%%%%%%%%%%%%%%%%%%%%%%%%%%%%%%%%%%%%%%%%%%%%%%%%%%%%%%%%%%%%%%%%%%%%%
%%                                                                 %%
%% Latex --专心内容创作,排版交给代码                              %%
%% 作者:曼妥思加可乐                                              %%
%% 参照Springer 官方模板修改                                       %%
%%                                                                 %%
%% 2021年11月18日                                                  %%
%%                                                                 %%
%%%%%%%%%%%%%%%%%%%%%%%%%%%%%%%%%%%%%%%%%%%%%%%%%%%%%%%%%%%%%%%%%%%%%

\documentclass[pdflatex,sn-mathphys]{sn-jnl}% Math and Physical Sciences Reference Style

\jyear{2021}%

%% as per the requirement new theorem styles can be included as shown below
\theoremstyle{thmstyleone}%
\newtheorem{theorem}{Theorem}%  meant for continuous numbers
%%\newtheorem{theorem}{Theorem}[section]% meant for sectionwise numbers
%% optional argument [theorem] produces theorem numbering sequence instead of independent numbers for Proposition
\newtheorem{proposition}[theorem]{Proposition}% 
%%\newtheorem{proposition}{Proposition}% to get separate numbers for theorem and proposition etc.

\theoremstyle{thmstyletwo}%
\newtheorem{example}{Example}%
\newtheorem{remark}{Remark}%

\theoremstyle{thmstylethree}%
\newtheorem{definition}{Definition}%

\raggedbottom
%%\unnumbered% uncomment this for unnumbered level heads

\begin{document}

\title[Article Title]{My First Latex Article}

\author*[1]{\fnm{Mentos} \sur{Cola}}\email{MentosCola@emails.com}

\author[1]{\fnm{Oreo} \sur{Milk}}\email{OreoMilk@emails.com}
\equalcont{These authors contributed equally to this work.}

\affil*[1]{\orgdiv{Mentos Office}, \orgname{Beijing University of Technology}, \orgaddress{\street{Street}, \city{Beijing}, \postcode{100124}, \country{China}}}

\abstract{This is my first time writing an article in Latex. The article is about the experiment that Mentos sugar added into Cola. Let's get started}

\keywords{Mentos, Cola}

\maketitle

\section{Introduction}\label{sec1}

\section{Mentos-Cola Test}\label{sec2}

\subsection{Test Principle}\label{subsec1}

\subsubsection{Coca-Cola or Pepsi}\label{subsubsec1}
Here is the main text

Case 01:

How many steps are needed to put Mentos sugar into Coke?
1. Open the lid of the Cola bottle
2. Put the Mentos in
3. Close the lid of the Cola bottle

Case 02:

How many steps are needed to put Mentos sugar into Coke?

1. Open the lid of the Cola bottle

2. Put the Mentos in

3. Close the lid of the Cola bottle

$\alpha$, $\beta$, $\gamma$

$\alpha$, $\beta$, $\gamma$

Mentos Cola.    %原字体

\textbf{Mentos Cola.}    %加粗

\underline{Mentos Cola.}    %下划线

\textit{Mentos Cola.}   %斜体

\underline{\textbf{\textit{Mentos Cola.}}}  %斜体加粗下划线

\begin{figure}[h]%
\centering
\includegraphics[width=0.5\textwidth]{fig01.jpg}        
%width=0.5\textwidth 页边距的0.5倍 图片文件名是fig01.jpg
\caption{I am Mentos Cola.}\label{fig01}
%图名是 I am Mentos Cola.    图片的标签是fig01 在文中引用时需要使用标签名
\end{figure}

\begin{table}[h]
\begin{center}
%居中
\caption{This is a table}\label{tab01}  
%添加表名和标签
\begin{tabular}{lll}
\hline
     & Mentos & Oreo \\ \hline
Cola & A      & B    \\
Milk & C      & D    \\ \hline
\end{tabular}
\end{center}
\end{table}

% 从Mathtype中复制得到的是  \[{x^2} - 2x + 1 = 0\]  删掉斜线和方括号
%行内公式
I am Mentos Cola.Today I am learning to solve ${x^2} - 2x + 1 = 0$. 

%行间公式
I am Mentos Cola.Today I am learning to solve equations.

\begin{equation}
{x^2} - 2x + 1 = 0 \label{eq01}
\end{equation}

It is so easy, $x = 1$.

Cite a equation \ref{eq01} 
Cite a reference \cite{bib1} 
Cite an image \ref{fig01}

\bibliography{sn-bibliography}% common bib file

\end{document}
